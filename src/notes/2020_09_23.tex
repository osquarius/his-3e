\subsection*{Historia ziem białoruskich}
\subsubsection*{Hołd carów Szujskich}
\begin{description}
    \item Polacy na Kremlu 1610 -- 1612 jako jedyna nacja w~historii
\end{description}
\subsubsection*{Strata wielu terenów przez Rzeczpospolitą}
\begin{itemize}
    \item powstanie Chmielnickiego na Ukrainie
    \item potop szwedzki
\end{itemize}
\subsubsection*{XVIII wiek}
\begin{description}
    \item Stanisław August Poniatowski --- ostatni król Polski
    \item w~1772 doszło do faktycznego rozbioru, a~w~1773 
    \item Tadeusz Rejtan --- poseł z~Nowogródka (na terenie Białorusi) --- jako przedstawiciel rdzennej ludności Białorusi przeciwstawił się I~rozbiorowi. Jest symbolem przyjaźni białorusko-polskiej
    \item II~rozbiór --- utracono Mińsk
    \item insurekcja kościuszkowska 1793 --- ostatni zryw niepodległościowy
\end{description}
\subsubsection*{Utworzenie administracji rosyjskiej na terenie Białorusi}
\begin{description}
    \item gubernie z~gubernatorami --- Rosja chciała zupełnie wchłonąć zajęte tereny I~RP i~wprowadzić tam swoją administrację i~swój język, wstrzymać rozwój kultury polskiej, litewskiej, białoruskiej oraz wytępić tożsamość narodową --- rusyfikacja
        \begin{itemize}
            \item cenzura
                \begin{itemize}
                    \item języka (również tych podobnych do rosyjskiego, czyli białoruskiego i~ukraińskiego)
                    \item literatury
                \end{itemize}
        \end{itemize}
\end{description}
\subsubsection*{Wincenty Konstanty Kalinowski --- bohater narodowy Polski, Litwy i~Białorusi}
\begin{itemize}
    \item reprezentant niechęci do rosyjskiego bata
    \item przeciwnik absolutyzmu carskiego
    \item listopad 2019 --- na polskim cmentarzu na Litwie na Rossie odbył się pochówek odnalezionych szczątków bohaterów narodowych, którzy dowodzili podczas powstania styczniowego, wśród nich Kalinowski
\end{itemize}
\subsubsection*{XX wiek}
\begin{description}
    \item I~wojna światowa
        \begin{itemize}
            \item bezpośrednia przyczyna: zabójstwo arcyksięcia Austro-Węgier Franciszka Ferdynanda przez serbskiego terrorystę
            \item wojna okopowa --- front przesuwa się zaledwie o~kilka metrów
            \item rok 1917
                \begin{itemize}
                    \item Rosja odstąpiła od wojny za sprawą burzliwych wydarzeń wewnątrz państwa:
                        \begin{itemize}
                            \item rewolucja lutowa
                                \begin{itemize}
                                    \item obalono carat
                                \end{itemize}
                            \item rewolucja październikowa
                                \begin{itemize}
                                    \item przejęcie władzy przez bolszewików --- komunistów z~Włodzimierzem Leninem na czele. Utrzymali się przy władzy do 1990 roku.
                                \end{itemize}
                        \end{itemize}
                        Rosja pogrążyła się w~wojnie domowej.
                    \item Wielka Rada Białoruska oraz I~Zjazd Wszechbiałoruski
                        \begin{itemize}
                            \item Białoruś miała stanowić odrębne państwo z~własnym językiem i~administracją
                        \end{itemize}
                \end{itemize}
            \item zakończenie wojny na wschodzie --- traktat brzeski 3~marca 1918. Niemcy wycofywali swoje wojska, ponieważ musieli skupić się na kampanii przeciwko zjednoczonym siłom Ententy (Francja, Wielka Brytania)
        \end{itemize}
\end{description}
\subsubsection*{Białoruska Republika Ludowa}
\begin{description}
    \item istaniała w~warunkach okupacji niemieckiej (mimo to Białorusini ogłosili niepodległość) od 25~marca 2018 do 1~stycznia 1919
    \item traktat ryski 1921
\end{description}
\subsubsection*{Wojna polsko-bolszewicka}
\begin{itemize}
    \item generał Stanisław Bułak-Bałachowicz wraz z~oficerami armii białoruskiej walczyli po stronie polskiej w~1920 roku
\end{itemize}
\subsubsection*{Symbole}
\begin{description}
    \item flaga niepodległej Białorusi --- biało-czerwono-biała --- nawiązywała do kolorów godła
    \item herb Białorusi komunistycznej --- zupełnie inna kolorystyka
\end{description}
\subsubsection*{II~wojna światowa}
\begin{description}
    \item agresja 17~września w~propagandzie sowieckiej --- żołnierz witany serdecznie, wyzwala
        \begin{description}
            \item faworyzowanie Białorusinów --- ZSRR usuwał z~urzedów Polaków, na ich miejsce pojawiali się
            \begin{itemize}
                \item Białorusini
                \item Żydzi
                \item Rosjanie z~ZSRR (przejmowali główne stanowiska)
            \end{itemize}
            \item przejęcie wielkiej własności ziemskiej --- szkoda dla Polaków, ponieważ była to dawna szlachta
            \item represje na duchownych --- katolicy i~prawosławni
            \item nacjonalizacja przedsiębiorstw --- Polacy i~Żydzi
        \end{description}
    \item czerwiec 1941 --- wejście Niemców na Kresy Wschodnie, w~tym na Białoruś
        \begin{description}
            \item część Białorusinów poparła Hitlera --- liczyli na to, że będą lepiej traktowani
                \begin{itemize}
                    \item stanowiska
                    \item autonomia
                \end{itemize}
            \item propaganda --- Słowianie to podludzie
            \item powstanie policji białoruskiej, formacji pomocniczych
            \item Niemcy oczekiwali, że Białorusini pomogą im w~zagładzie Żydów
            \item duże wywózki na roboty do Rzeszy
            \item aż 25\% populacji Białorusi zginęło w latach 1939 -- 1945
        \end{description}
\end{description}
\subsubsection*{Porozumienie białowieskie 8~grudnia 1991 --- rozpad ZSRR}
\begin{itemize}
    \item powstała niepodległa Białoruś i~wróciły historyczne barwy
    \item w~1994 roku wybory wygrał Aleksander Łukaszenka
    \item powróciła sowiecka ideologia w~wyniku referendum w~1995 (sfałszowane)
    \item godło socjalistyczne
\end{itemize}
