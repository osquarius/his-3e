\subsection*{Wojna o~Górski Karabach}
\subsubsection*{Czym jest Górski Karabach}
\begin{description}
    \item na terenie Azerbejdżanu
    \item obszar zamieszkały przez Ormian
    \item obszar ok \(\frac{1}{2}\) wielkości województwa świętokrzyskiego
    \item 150 tys. mieszkańców
    \item obecnie (od 1992) samozwańcze państwo, którego nie uznaje nikt na świecie
    \item geneza:
        \begin{itemize}
            \item Górski Karabach był kiedyś częścią Armenii, która w~średniowieczu była bardzo rozległym państwem, ale zmieniła później granice
            \item XIX wiek --- ekspansja Turcji i~Rosji spowodowała podział Armenii
            \item XX wiek --- ZSRR --- w~tym okresie Armenia i~Azerbejdżan miały status republik radzieckich
            \item Ormianie --- indoeuropejczycy, chrześcijanie (jeden ze starszych narodów chrześcijańskich, najstarsze państwo, które przyjęło chrześcijaństwo --- w~IV wieku, wcześniej niż Rzym!)
            \item Azerowie (język turecki) --- islam
            \item w~czasach ZSRR Górski Karabach znalazł się na terenie Azerbejdżańskiej Republiki Radzieckiej, ale otrzymał status Obwodu Autonomicznego
            \item 1988 --- początek walk Azersko-Ormiańskich o~Górski Karabach
            \item 1991 --- rozpad ZSRR \(\implies\) niepodległa Armenia i~Azerbejdżan, Górski Karabach stracił swój status Obwodu Autonomicznego (powinien być częścią Azerbejdżanu, ale tak się nie stało)
                \begin{itemize}
                    \item Ormianie z~Górskiego Karabachu nie uznali władzy Azerbejdżanu i~stawili zbrojny opór
                    \item do 2020 spokój, a~później znowu walki z~Turcją i~Rosją w~tle
                    \item symboliczna flaga --- flaga Armenii z~wydzielonym obszarem oznaczającym oddzielenie Górskiego Karabachu od Armenii
                \end{itemize}
        \end{itemize}
\end{description}